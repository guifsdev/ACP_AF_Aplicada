%!TEX root = ../main/main.tex

\section*{Resumo}

O desenvolvimento de modelos de mensuração da qualidade de cursos superiores e instituições ainda é assunto bastante discutido na literatura internacional dada a diversidade de dimensões e interesses de avaliação. Organismos multilaterais têm estimulado a avaliação como instrumento de aprimoramento e controle social. No que se refere às políticas públicas do Brasil, houve um considerável aumento de gastos em ensino, sobretudo na última década. Nacionalmente, uma das métricas de mensuração da qualidade do ensino superior e das instituições é o Enade, com provas realizadas em ciclos trienais de avaliação. Este trabalho tem como principal objetivo investigar os dados de avaliação de 2015 disponibilizados pelo INEP e verificar a composição do Conceito Preliminar de Cursos (CPC) como indicador de qualidade. Com este intuito foram utilizadas três técnicas estatísticas: Análise Exploratória de Dados, Análise do Componente Principal e Análise do Fator. Para diferentes configurações de amostragem, os resultados indicam que a Organização Didático-pedagógica, Infraestrutura e Instalações e Oportunidades de Ampliação da Formação são as variáveis que mais exercem influência na composição do CPC.

\textbf{Palavras-chave:} Enade, Conceito Preliminar de Cursos, Qualidade do ensino superior.

\section*{Abstract}

The development of models for measuring the quality of higher education and institutions is still very much discussed in the international literature given the multiplicity of evaluation optics and interests. Multilateral organizations have stimulated evaluation as an instrument of improvement and social control. With regard to public policies in Brazil, there has been a considerable increase in education spending, especially in the last decade. Nationaly, one of the higher education quality measurement metrics is Enade with its trienal exams. The main objective of this study is to investigate the evaluation data of 2015 made available by INEP and verify if the Preliminary Course Concept is an indicator of quality. In this regard, three statistical methodologies has been used: Exploratory Data Analysis, Principal Component Analysis and Factor Analysis. Through different sampling arrangements, results indicate that Didactic pedagogical Organization, Infrastructure and Facilities and Opportunities for Expansion of Training are variables that influence the most in the composition of the Preliminary Course Concept.

\textbf{Keywords:} Enade, Preliminar Course Concept, Higher education quality.
\pagebreak