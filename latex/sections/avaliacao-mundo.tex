%!TEX root = ../main/main.tex

\section{AVALIAÇÃO DO ENSINO SUPERIOR NO\\MUNDO}

\citeonline[p. 417]{Costa_Souza_Ramos_Silva_2012} destacam algumas características inerentes ao setor de produção educacional. São elas: i) a natureza múltipla e intangível do produto – os produtos educacionais podem ser classificados como: conhecimento e habilidades, valores, atitudes, entre outras características; ii) a participação do cliente no processo produtivo – o cliente (aluno) não é meramente um demandante da mercadoria, mas atua de forma decisiva no processo produtivo; iii) a heterogeneidade dos serviços – devido à participação do estudante no processo produtivo, as unidades produtivas se diferenciam umas das outras; iv) a dimensão temporal – os resultados obtidos no processo produtivo podem não ser suficientes para uma mensuração completa da produção do setor educativo, visto que é necessário observar uma trajetória completa da vida dos estudantes; v) o caráter acumulativo do ensino; vi) a incidência de fatores exógenos – essa característica tem como embasamento a denominada educação informal, que não é obtida pelos anos de estudos, mas sim por experiências fora do setor educacional. 

O tema sobre como devem ser alocados os recursos públicos no setor de educação superior vem direcionando a grande maioria dos estudos para a mensuração da eficiência das IES públicas. Ao longo dos anos, muitos estudos têm como objetivo mensurar a eficiência e ranquear as IES públicas através de seu grau de eficiência. Ademais, cada país tem sua estrutura de financiamento e alocação de recursos que serve como base para a estimação da eficiência do setor educacional superior. \Ibidem[p. 418]{Costa_Souza_Ramos_Silva_2012}

Os métodos mais utilizados para medir a eficiência dentro do contexto do setor educacional são os paramétricos e os não paramétricos. As técnicas estatísticas empregadas são baseadas nos Mínimos Quadrados Ordinários (MQO) de regressão para análise de fronteira estocástica. \cite[p.~418]{Costa_Souza_Ramos_Silva_2012}.

Por outro lado, como \citeonline[p.~1]{Bernhardt_1998} afirma, a mensuração das unidades escolares com base em indicadores múltiplos pode não ser suficiente para o aprimoramento da unidade de ensino. De acordo com a autora, isso chegar a ser uma forma enganosa de se compreender o desempenho.

Deste modo, qualquer avaliação de desempenho que envolva mais de uma variável deve incluir em meio a estas, outras variáveis relacionadas a aspectos adicionais. São elas: dimensão demográfica, percepções, aprendizado do estudante e processos escolares. São esses fatores que fornecem as respostas que necessitam para melhorar os resultados. \Ibidem{Bernhardt_1998}.
