%!TEX root = ../main/main.tex

\section{CONCLUSÃO}

Apesar de a educação ser um fator essencial para o desenvolvimento de uma nação, o Brasil ainda desponta como uma das nações com maior proporção de adultos que não tiveram a educação primária \cite[p.~44]{OCDE_2017}. Por esse motivo, acredita-se que as contribuições ao tema, por menores que sejam, são valiosas. Ainda que se reconheça a relevância das fases de base da educação, bem como a variedade de instrumentos de avaliação das mesmas, buscou-se neste trabalho analisar a sistemática de avaliação do ensino superior. 

Procurou-se, fornecer um instrumento informativo, possivelmente fundamentador de tomada de decisão aos atores responsáveis, direta e indiretamente pela gestão das unidades de ensino superior e cursos de graduação. À comunidade acadêmica das IES, foram apresentados os principais elementos que devem ser estimulados, desenvolvidos, e mantidos de forma a lograr qualidade na avaliação do Enade.

As aplicações de metodologias estatísticas deram sustentação ao desenvolvimento de respostas às perguntas e hipóteses previamente formuladas para a elaboração desta pesquisa. Neste sentido, a primeira das principais apurações é a de que a dimensão do Enade que incorpora a opinião do estudante incide fortemente na formação do CPC. Nessa perspectiva, acredita-se que existem dados de dimensões e variáveis latentes que ainda podem ser extraídos.

Verificou-se que a qualidade do ensino dos cursos de graduação resulta, principalmente:
\begin{enumerate}[label=\roman*)]
\item do adequado ajustamento da estrutura curricular levando em consideração a utilidade dos conteúdos ao exercício da profissão;
\item do ambiente onde se desenvolvem as atividades de ensino; 
\item dos estímulos a continuidade dos estudos após a graduação;
\item da quantidade de mestres e doutores presentes dos cursos.
\end{enumerate}

Em relação à primeira dimensão, é possível afirmar que fatores como coerência da estrutura curricular com os objetivos dos cursos; adequação e atualização das ementas e das disciplinas; uma seleção de conteúdos satisfatória; adequação, atualização e relevância da bibliografia são favoráveis à melhora do indicador de qualidade do curso de graduação.

No que se refere à Infra-estrutura e Instalações podem ser citados elementos como: ambientes projetados para atender a todos os requisitos necessários para a realização das atividades de ensino; que levem em consideração o conforto não apenas dos discentes, mas de toda comunidade acadêmica e que sejam atentem para questões relacionadas à acessibilidade. Aliado a estes, sistema de segurança, iluminação, ventilação, equipamentos e mobiliários adequados, podem impactar de forma significativa no CPC.

A promoção de estímulos à ampliação da formação dos discentes são fatores igualmente importantes na composição do CPC. Devem ser criados meios para que, durante o período da graduação, sejam apresentados os caminhos e possibilidades de prosseguimento dos estudos aos alunos. É seguro afirmar que os esforços seriam irrisórios quando comparados aos resultados favoráveis oriundos de avanços e contribuições em relação ao desenvolvimento de pesquisas.

Finalmente, aos desenvolvedores de políticas educacionais, busca-se por meio desta material viabilizar conteúdo informacional para que seja promovida a adequada ampliação do ensino superior pelas diversas classes da sociedade sem, contudo, negligenciar sua qualidade.