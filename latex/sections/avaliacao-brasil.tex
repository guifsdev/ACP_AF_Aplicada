%!TEX root = ../main/main.tex

\section{AVALIAÇÃO DO ENSINO SUPERIOR NO\\BRASIL}
Na literatura em torno da avaliação do ensino superior no Brasil é comum constatar que diversos autores destacam a complexidade deste sistema e os diversos mecanismos que orbitam ao seu redor. Não obstante, o Ministério da Educação por meio do INEP publica com relativa frequência as notas técnicas e portarias onde são descritas de forma esmiuçada novas metodologias e mudanças aplicadas na avaliação do ensino superior. O que se observa é uma permanente tentativa do governo em aprimorar os critérios e métricas em vias de racionalizar a mensuração da qualidade do ensino superior no país. Nessa perspectiva, o objetivo desta sessão é apresentar uma visão dos principais aspectos da sistemática de avaliação das instituições e cursos superiores pela abordagem das principais dificuldades na formulação de indicadores e métricas, bem como um panorama evolutivo dos processos de avaliação.

\subsection{Breve panorama evolutivo das metodologias de avaliação}
	\citeonline[p.~318]{Barbosa_Freire_Crisostomo_2011} classificam a avaliação das instituições de ensino superior (IES) em externa e interna e evidenciam que estas têm estado no centro dos debates tanto nacional quanto internacionalmente. Tal fato tem relação com a percebida necessidade de otimização de recursos humanos e materiais das universidades. Evidentemente, os processos de avaliação fazem uso extensivo de indicadores. De acordo com esses autores, apesar de serem muito criticados no Brasil, são instrumentos de firme aceitação pelos governos que parecem não abandoná-los. \Ibidem[p.~319]{Barbosa_Freire_Crisostomo_2011}.

	\citeonline[p.~385]{Zandavalli_2009} aponta a influencia do banco mundial no desenvolvimento de políticas públicas que subsidiam o entendimento do sistema de avaliação da educação superior. De forma geral, é possível constatar pela análise da literatura disponível, os instrumentos de avaliação em sua concepção e finalidade, têm certa influência estrangeira. Como esse autores de forma bem íntegra analisam, as movimentações iniciais pela reestruturação do ensino superior no Brasil coincidem com as mobilizações de universitários contrários ao regime militar de 1960. \Ibidem[p.~386]{Zandavalli_2009}.
	\pagebreak

	Neste mesmo ano tem grande destaque o plano Atcon\footnotemark~ como um dos primeiros processos voltados a avaliar a estrutura das universidades brasileiras com a proposta de associar investimentos e resultados \cite[p.~320]{Barbosa_Freire_Crisostomo_2011}. Esse plano se estabeleceu como um marco de primeira prática de avaliação do ensino durante o regime militar pela adequação da educação superior ao  modelo econômico capitalista \cite[389]{Zandavalli_2009}. As instituições brasileiras foram satisfatoriamente beneficiadas pelos estudos e propostas do Plano Atcon. Uma de suas recomendações que mais repercutiram a sua tentativa de gerar uma correspondência entre investimentos e resultados parece ser o quadro-organograma e os \textit{decision-makers} da ``universidade empresa''. \Ibidem[p.~390]{Zandavalli_2009}.
	\footnotetext{Rudolph Atcon foi um consultor norte americano responsável pelo estudo para reformulação das estruturas das universidades brasileiras. Seu trabalho se deu com base em visita a 12 universidades e a procura por fatores que refletissem a perspectiva de modernização. Seu estudo se deu om base nos pressupostos americanos de racionalidade eficiência e eficácia das instituições. Deu origem ao documento ``Rumo à Reformulação Estrutural da Universidade Brasileira'' com as propostas para reformulação do ensino superior no país. \cite[387]{Zandavalli_2009}}

	A década de 1970 foi marcada em sua maior parte pela operacionalização progressiva dos planos traçados pelo Plano Atcon em 60. Em vista disso, esse período se mostrou relativamente apagado em termos de novas propostas de reformulação. O Plano Atcon, embora seja um marco pela sua importância na concretização de um considerável avanço em termos de adequação da estrutura do ensino no Brasil, carecia do envolvimento e diálogo com os professores e a sociedade civil. \Ibidem[p.~401]{Zandavalli_2009}. Em decorrência disso, em 1980 houve a implantação do Programa de Avaliação da Reforma Universitária (PARU), onde passou-se a considerar as propostas da sociedade e dos discentes.

	De acordo com \citeonline[p.~402]{Zandavalli_2009}, o PARU contou com uma participação mais significativa de pesquisadores e professores universitários quando de sua criação e aplicação -- condições fundamentais para reformulação da educação superior. O autor destaca que as principais demandas situavam-se em torno da democratização e acesso ao ensino pelas diferentes classes da sociedade e nos princípios básicos das universidades: formação de profissionais, produção e disseminação de conhecimentos. \apud[p.~403]{Almeida_Junior_2004}{Zandavalli_2009}.

	A avaliação do ensino superior volta ao centro do cenário político na década de 80, no governo Sarney, momento em que a grande escassez de recursos públicos impulsionava os debates na Comissão Nacional para Reformulação do Ensino Superior (CNRES) (Decreto no 91.117/1985) e do Grupo Executivo para a Reformulação do Ensino Superior (GERES) (Portaria no 100/1986). As propostas levantadas por esses dois grupos de trabalho apresentam pontos em comum em relação às instituições de ensino superior (IES): implantação de sistemas de avaliação e valorização do desempenho de universidades e necessidade de melhoria no processo de financiamento pelo aprimoramento da gestão do uso de recursos públicos. \cite[p.~321]{Barbosa_Freire_Crisostomo_2011}.

	A partir de então, o estabelecimento de estruturas de controle baseadas em indicadores tornou-se uma necessidade da gestão pública. Contudo, apenas em 1993 é criado o Programa de Avaliação das Universidades Brasileiras (PAIUB) que introduzia, dentre outras, as propostas de indicadores de avaliação das IES; verificação em nível nacional; carência de eficiência no financiamento e de prestação de contas aos financiadores. O programa também aponta a necessidade de verificação de dimensões quantitativas e qualitativas em consonância com a frequência de processos de verificação de desempenho. É reforçada a necessidade de criação de um sistema de controle de avaliação de desempenho institucional e a qualidade dos cursos coordenado pela Secretaria de Educação Superior (SESu) do Ministério da Educação – MEC e envolvimento da comunidade acadêmica. \Ibidem[p.~322]{Barbosa_Freire_Crisostomo_2011}.

	De acordo com \citeonline{Barbosa_Freire_Crisostomo_2011}, neste mesmo período, o estabelecimento dos indicadores das IES se deu paralelamente a caracterização das dimensões das universidades, quais sejam avaliação de cursos, avaliação de alunos, avaliação de professores, avaliação didático-pedagógica do ensino, avaliação de servidores técnico-administrativos e avaliação das carreiras. O critério de organização da avaliação com base em dimensões é um dos fundamentos do Sistema Nacional de Avaliação da Educação Superior (SINAES – Lei 10.861/2004) como será visto mais à adiante.

\subsection{Dificuldades e necessidades em avaliar IES}
A avaliação da qualidade de cursos superiores ou instituições de ensino é um processo inerentemente difícil pela diversidade de perspectivas e interesses. De forma ilustrativa, o gestor de recursos públicos poderia voltar sua atenção ao custo por aluno. O estudante, por outro lado, poderá valorizar mais a empregabilidade após formado -- dimensão esta quase sempre negligenciada pelo primeiro observador. O professor universitário, por sua vez, poderia voltar sua preocupação a quantidade de alunos em sala de aula ou mesmo às condições de estabilidade. 

Como é possível observar, a aplicação de mecanismos de avaliação capazes de aferir de forma eficiente a qualidade das IES constitui tarefa trabalhosa. De acordo com \citeonline[p.~322]{Barbosa_Freire_Crisostomo_2011} o tema está em constante discussão no Brasil. Para estes autores, outro fator que contribui para a dificuldade do processo de avaliação é a distinção entre a avaliação do ensino superior e avaliação de instituições públicas. A primeira é aplicável tanto às instituições públicas quanto privadas. A avaliação de instituições públicas, por sua vez, deve se preocupar, concomitantemente à qualidade do ensino, a eficiência na alocação dos recursos públicos, uma vez que necessita realizar a prestação de contas para com a sociedade. 

Historicamente no Brasil há uma concentração maior do ensino superior nas instituições de ensino públicas \cite{Barbosa_Freire_Crisostomo_2011}, realidade que pode estar relacionada à inexistência de pagamento de mensalidades pelos alunos e a proliferação de instituições particulares com cursos de baixa qualidade. Isso gera necessidade ao governo de prestar atenção especial às políticas de ampliação de acesso ao ensino superior bem como criar mecanismos de avaliação, auditoria e gestão das IES de maneira geral.

A subjetividade envolvida no processo de avaliação de IES torna necessária a utilização de sistemas de avaliação de desempenho aplicados nacionalmente. 
O desempenho alcançado em tais avaliações passou a figurar como parâmetro de balizamento da qualidade de cursos e instituições. Este é um dos fatores que contribuem tanto para o desenvolvimento de um mercado da educação, quanto para necessidade de se determinar a conformidade dos sistemas baseados em avaliação para emitir juízo de valor acerca da qualidade. \cite[p.~108]{Bertolin_Marcon_2015}.

De acordo com \citeonline[p.~318]{Barbosa_Freire_Crisostomo_2011} a avaliação desempenho baseada em exames é motivada tanto no Brasil quanto internacionalmente pela responsabilidade institucional para com os \textit{stakeholders} (ou financiadores), principalmente no caso de IES públicas. Por outro lado \citeonline[p.~389]{Zandavalli_2009} aponta que na realidade, o que se busca com os dados estatísticos confiáveis é um melhor domínio das IES a fim de acabar com a gratuidade no ensino superior e criar uma lógica de mercado ao sistema universitário.

Sob uma dimensão funcional, a dificuldade em avaliar as IES decorre do fato de estas serem entidades muito distintas das empresas. As IES apresentam uma complexidade muito mais acentuada pela existência de peculiaridades distintas das organizações empresariais uma vez que têm como objetivo formar profissionais capazes de exercer atividades de ensino, pesquisa e extensão. Deste modo, as dificuldades estão concentradas na mensuração da qualidade destas atividades. \cite[p.~318]{Barbosa_Freire_Crisostomo_2011}.

A dificuldade em avaliar IES gira em torno da diversidade de grupos de interesse com diferentes graus de subjetividade e visões distintas sobre qualidade em educação. Em vista disso as avaliações tem sido concebidas e implementadas de formas diversas tanto no Brasil quanto em outros países. Apesar disso, organismos multilaterais têm estimulado em nível mundial a criação de mecanismos de avaliação como forma de maximização de benefícios e garantias sociais. \cite[p.~106]{Bertolin_Marcon_2015}

Fatores importantes na avaliação de IES são métricas e indicadores adequados. Segundo \citeonline[p.~319]{Barbosa_Freire_Crisostomo_2011} as métricas e indicadores em uso no Brasil têm sido alvo de muitas críticas e questionamentos. Neste sentido, para o autor é necessário que se façam constantes avaliações no sentido de aperfeiçoá-los. Todavia, é visível que há um certo esforço dos governos em não abandoná-los.

A visualização do desempenho de cursos e IES a partir dos indicadores parecer ser a forma mais coerente de se trabalhar com métricas para realizar a mensuração e acompanhamento da qualidade do ensino. Apesar disso, existem dimensões que encontram-se não abrangidas por tais métricas. Uma dessas dimensões é abordada por \citeonline[p.~114]{Bertolin_Marcon_2015} e faz referência ao \textit{background} dos alunos. Este elemento pode ser entendido como condições relacionadas ao contexto familiar e situação sócio-econômica dos mesmos. Neste contexto, o autor faz referência ao capital cultural proposto por Bordieu e aborda os elementos que impactam diretamente nas condições de aprendizagem dos alunos e consequentemente seu desempenho em exames de grande abrangência.

O entendimento deste ``pano de fundo'' dos estudantes como fator determinante no desempenho em exames faz surgir controvérsias acerca da validade dos exames de mensuração de desempenho de ensino superior. Alunos com melhores panos de fundo são aqueles nascidos em famílias com melhores condições financeiras. Podem estudar sem se preocupar com provisões e tem melhores condições de aprendizagem pelo fato de os pais terem tido maior tempo de estudo e, portanto serem capazes de transmitir como mais facilidade o conhecimento aos seus filhos.

%\cite[p.~]{Bertolin_Marcon_2015}
%\cite[p.~]{Barbosa_Freire_Crisostomo_2011}
%\cite[p.~]{Zandavalli_2009}


%\subsection{O SINAES e o ENADE}
%\subsection{Os indicadores}
%\subsection{Métricas}
\pagebreak











